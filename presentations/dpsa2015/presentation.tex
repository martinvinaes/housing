\documentclass[aspectratio=169]{beamer}

%
% Choose how your presentation looks.
%
% For more themes, color themes and font themes, see:
% http://deic.uab.es/~iblanes/beamer_gallery/index_by_theme.html
%
\mode<presentation>
\mode<presentation>
{
  \usetheme[compress]{Ilmenau}      % or try Darmstadt, Madrid, Warsaw, ...
  \usecolortheme{beaver} % or try albatross, beaver, crane, ...
  \usefonttheme{default}  % or try serif, structurebold, ...
  \setbeamertemplate{navigation symbols}{}
  \setbeamertemplate{caption}[numbered]
} 
\usepackage[english]{babel}
\usepackage[utf8x]{inputenc}
\usepackage{graphicx}
\usepackage{setspace}
\usepackage{tikz} \usetikzlibrary{snakes}
\usepackage{epstopdf}
\usepackage[round]{natbib}
\setbeamertemplate{enumerate items}[circle]
\setbeamertemplate{itemize items}[circle]
\setbeamertemplate{blocks}[rounded][shadow=false] 
\usepackage{remreset}% tiny package containing just the \@removefromreset command
\makeatletter
\@removefromreset{subsection}{section}
\makeatother
\setcounter{subsection}{1}
\setbeamertemplate{section in toc}[sections numbered]


\usepackage[english]{babel}
\usepackage[utf8x]{inputenc}
\title[Housing bubbles]{Housing Bubbles and \\Support for Governing Parties}
\author{\textbf{Frederik G. Hjorth} \and \textbf{Martin V. Larsen}}
\institute[UCPH]{\small Department of Political Science \\ University of Copenhagen}
\date[Octoboer 2015]{ 47th annual meeting of the DPSA \\ October 30th 2015}


\begin{document}
	
	\begin{frame}
		\titlepage
	\end{frame}
	
	
\section{Politicized homes?}

	\begin{frame}
	%{Politicized homes}
	Key economic feature of post-industrial societies: mass home ownership.
	
	\vspace{0.2in} 

	\begin{itemize}
		\item main form of capital ordinary people have.
		\item key part of one's control over one's immediate context.
		\item often the focus of political rhetoric.
	\end{itemize}
	
		\vspace{0.2in} \pause
		
	$\rightsquigarrow$ but: relatively little research on how home ownership shapes political behavior.
		\end{frame}	
	
		\begin{frame}
		\citet{ansell2014political}: house price appreciation reduces preferences for social insurance. 
		
		$\rightsquigarrow$ see also \citet{di2007formation}.
						
		\vspace{0.3in}	\pause
						
		Here: \emph{local house prices} and \emph{incumbent government support}	.					
		
		\vspace{0.3in} \pause	
		
		Feeds into literature on effect of local economy on incumbent support.
       			
		\end{frame}	
		
		
	\begin{frame}

	Slide on extant literature.	

			
	\end{frame}	
	
	\begin{frame}
			
	Note of caution...	
			
	We do not know what the mechanism is. Simply, establish an effect.
	\end{frame}	
	
\section{Empirical setting}	
\begin{frame}
In international comparison, DK's housing bubble exceptionally volatile:

\begin{center}
\includegraphics<1>[width=0.8\textwidth]{../../figures/intcomparison.png}
\end{center}
\end{frame}	
	
	\begin{frame}
However, still great variation within DK (municipalities).

	\begin{center}
\includegraphics<1>[width=0.9\textwidth]{../../figures/manylines_oneplot.eps}	
		\end{center}
	\end{frame}	
	
	\begin{frame}
		However, still great variation within DK (municipalities).
		
		\begin{center}
			\includegraphics<1>[width=0.9\textwidth]{../../figures/manylines_oneplot.eps}	
		\end{center}
	\end{frame}	
	
	
\section{Data}	
\begin{frame}
%{Data on house prices I}
House prices are hard to measure
\begin{itemize}
\item for individual home, unobserved outside of purchase/sale \pause
\item plausible proxy: realized house prices in local context \pause

$\rightsquigarrow$ context effect \emph{and} personal experience
\end{itemize}
% - no one knows exactly what a house costs until it is on the market - but good proxy is how much houses in the same area costs. 

\vspace{0.2in} \pause

Data on municipal selling prices from the Danish Mortgage Banks' Federation covering twenty years (1992:2013). 
%Covers important housing-bubble. 

%\vspace{0.2in}
%The data is on the municipal level and covers the average selling price of houses and condos in the municipality in a given quarter (mean number of sales per quarter is 112).

%\vspace{0.2in}
%This gives us a good indicator of what we want to know; namely, whether home-owners should expect the value of their house (or condo) to have appreciated or depreciated. 

\end{frame}	

\begin{frame}
%{Data on house prices II}
\begin{center}
\includegraphics<1>[width=0.9\textwidth]{../../figures/priceacrossmuni.eps}
\includegraphics<2>[width=0.9\textwidth]{../../figures/manylines_oneplot.eps}	
\includegraphics<3>[width=0.9\textwidth]{../../figures/prices_histogram.eps}	
\pause[3]

$\mu=0.05$ \hspace{0.1in} $\sigma=0.11$
\end{center}
\end{frame}	

\begin{frame}

\end{frame}	

\begin{frame}

\end{frame}	


\section{Results}
\begin{frame}
	\footnotesize
		\only<1>{\begin{table}[htbp]\centering
\def\sym#1{\ifmmode^{#1}\else\(^{#1}\)\fi}
\caption{Estimated effects of house prices on electoral support for governing parties.} \label{tab1}
\begin{tabular}{l*{5}{c}}
\hline\hline
                    &\multicolumn{1}{c}{(1)}        &\multicolumn{1}{c}{(2)}        &\multicolumn{1}{c}{(3)}        &\multicolumn{1}{c}{(4)}        &\multicolumn{1}{c}{(5)}        \\
\hline
$\Delta$ house price&        0.10\sym{**}&        0.12\sym{**}&        0.05\sym{**}&        0.05\sym{**}&        0.01\sym{*} \\
                    &      (0.01)        &      (0.01)        &      (0.01)        &      (0.01)        &      (0.01)        \\
[1em]
\hline Precinct FE  &                    &$\checkmark$        &$\checkmark$        &$\checkmark$        &$\checkmark$        \\
[1em]
Year FE             &                    &                    &$\checkmark$        &$\checkmark$        &$\checkmark$        \\
[1em]
Year FE * Structural factors&                    &                    &                    &$\checkmark$        &$\checkmark$        \\
[1em]
Year FE * Municipality FE&                    &                    &                    &                    &$\checkmark$        \\
\hline
Observations        &        4192        &        4192        &        4192        &        4170        &        4170        \\
RMSE                &        8.40        &        7.16        &        5.71        &        4.77        &        2.84        \\
\hline\hline
\multicolumn{6}{l}{\footnotesize Standard errors in parentheses}\\
\multicolumn{6}{l}{\footnotesize \sym{*} \(p<0.05\), \sym{**} \(p<0.01\)}\\
\end{tabular}
\end{table}
}
		\only<2>{\begin{table}[htbp]\centering
\def\sym#1{\ifmmode^{#1}\else\(^{#1}\)\fi}
\caption{Estimated effects of house prices on electoral support for governing parties at t+1.} \label{tab2}
\begin{tabular}{l*{5}{c}}
\hline\hline
                    &\multicolumn{1}{c}{(1)}        &\multicolumn{1}{c}{(2)}        &\multicolumn{1}{c}{(3)}        &\multicolumn{1}{c}{(4)}        &\multicolumn{1}{c}{(5)}        \\
\hline
$\Delta$ house price&        0.12\sym{**}&        0.14\sym{**}&       -0.02        &       -0.01        &        0.02        \\
                    &      (0.01)        &      (0.01)        &      (0.01)        &      (0.01)        &      (0.01)        \\
[1em]
\hline Precinct FE  &                    &$\checkmark$        &$\checkmark$        &$\checkmark$        &$\checkmark$        \\
[1em]
Year FE             &                    &                    &$\checkmark$        &$\checkmark$        &$\checkmark$        \\
[1em]
Year FE * Structural factors&                    &                    &                    &$\checkmark$        &$\checkmark$        \\
[1em]
Year FE * Municiplaity FE&                    &                    &                    &                    &$\checkmark$        \\
\hline
Observations        &        3227        &        3227        &        3227        &        3209        &        3209        \\
RMSE                &        8.62        &        7.11        &        6.22        &        5.24        &        3.05        \\
\hline\hline
\multicolumn{6}{l}{\footnotesize Standard errors in parentheses}\\
\multicolumn{6}{l}{\footnotesize \sym{*} \(p<0.05\), \sym{**} \(p<0.01\)}\\
\end{tabular}
\end{table}
}
		\only<3>{\begin{table}[htbp]\centering
\def\sym#1{\ifmmode^{#1}\else\(^{#1}\)\fi}
\caption{Estimated effects of house prices on electoral support for governing parties at t-1.} \label{tab3}
\begin{tabular}{l*{4}{c}}
\hline\hline
                    &\multicolumn{1}{c}{(1)}        &\multicolumn{1}{c}{(2)}        &\multicolumn{1}{c}{(3)}        &\multicolumn{1}{c}{(4)}        \\
\hline
$\Delta$ house price&       -0.03\sym{**}&       -0.04\sym{**}&        0.07\sym{**}&        0.08\sym{**}\\
                    &      (0.01)        &      (0.01)        &      (0.01)        &      (0.01)        \\
[1em]
\hline Precinct FE  &                    &$\checkmark$        &$\checkmark$        &$\checkmark$        \\
[1em]
Year FE             &                    &                    &$\checkmark$        &$\checkmark$        \\
[1em]
Year FE * Structural factors&                    &                    &                    &$\checkmark$        \\
\hline
Observations        &        4197        &        4197        &        4197        &        4173        \\
RMSE                &        8.80        &        7.50        &        6.46        &        5.04        \\
\hline\hline
\multicolumn{5}{l}{\footnotesize Standard errors in parentheses}\\
\multicolumn{5}{l}{\footnotesize \sym{*} \(p<0.05\), \sym{**} \(p<0.01\)}\\
\end{tabular}
\end{table}
}
\end{frame}		

\begin{frame}
\begin{center}
	\includegraphics<1>[width=0.6\textwidth]{../../figures/posnegplot}
	\includegraphics<2>[width=0.6\textwidth]{../../figures/volaplot}
\end{center}

\end{frame}	

\section{Discussion}

	\begin{frame}
	%{What is going on?}
	We have shown evidence suggesting that falling - but not rising - house prices hurt incumbents in Denmark.
	
	\vspace{0.2in} \pause
	
	\begin{itemize}
		\item is this convincing?
		\item what other analyses would you like to see?
		\item which interesting (theoretical and real world) implications do you think this has?
		\item do you think this is a case where personal economic grievances matter? 
	\end{itemize}	
		
	\end{frame}		

\begin{frame}
	\bibliography{litliste}
		\bibliographystyle{apsr}
\end{frame}


\end{document}